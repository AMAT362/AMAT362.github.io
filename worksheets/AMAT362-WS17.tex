\documentclass[addpoints,12pt]{exam}
\usepackage{fullpage,xypic,mathrsfs}
\usepackage{amsmath, amsthm, amssymb, latexsym}
\usepackage{enumerate}
\usepackage{hyperref}
%\usepackage[sans]{dsfont}

\usepackage{graphicx}

\usepackage[margin=0.7in]{geometry}

\usepackage{bm}
\newcommand{\mbf}[1]{\boldsymbol{\mathbf{#1}}}

\title{\vspace{-1in} AMAT 362---Work Sheet 17}
\date{Due: April 6th, 2022. Worth \numpoints\ points.}
\author{Dr.~Justin M. Curry}


%\input{shortcuts}

\begin{document}
\maketitle

% \begin{center}
% \fbox{\fbox{\parbox{6in}{\centering
% \textbf{Open Book. Open Notes. Open Friends.}}}}
% \end{center}
% \vspace{0.2in}
\makebox[.9\textwidth]{Name:\enspace\hrulefill}
% % \vspace{0.2in}
% % \makebox[\textwidth]{Instructor’s name: Dr. Justin Curry}





\begin{questions}

\question[1] Suppose a number is generated uniformly at random from the unit interval $(0,1)$, i.e. $X\sim \text{Unif}(0,1)$. What's the probability of $X$ being within 2 decimal places of .35, after rounding? For example, .349 would round up to .35 and .353 would round down.

\vspace{1.2in}

\question[1] Repeat the question above except under the assumption that $X\sim \text{Norm}(0,1)$ distribution, i.e. normally distributed with mean 0 and variance 1.

\vspace{1.2in}

\question[5] Suppose $X$ is a random variable whose density is $f(x)=cx(1-x)$ for $0<x<1$ and $f(x)=0$ otherwise.

\noaddpoints
\begin{parts}

\part[1] Find the value of $c$ in order for this to be a valid PDF.
\vspace{1.2in}
\part[1] $P(X\leq 1/2)$
\vspace{1.2in}
\part[1] $P(X\leq 1/3)$
\vspace{1.2in}
\part[2] Make a drawing of the function $F(s):=\int_{-\infty}^s f(x) dx$, which is called the \emph{cumulative distribution function (CDF)} of $X$.
\vspace{2in}
\end{parts}
\addpoints

\question[4] \emph{We'll do this in class, but please copy down the derivations and understand them!}
Recall that $X$ is exponentially distributed with parameter $\lambda$, written $X\sim \text{Exp}(\lambda)$, if its PDF is $f_X(x)=\lambda e^{-\lambda x}$ for $x\geq 0$ and is zero otherwise.

 \noaddpoints
\begin{parts}
\part[1] Derive a formula for $E(X)$ in terms of $\lambda$.
\vspace{1.5in}
\part[1] Derive a formula for $V(X)$ in terms of $\lambda$.
\vspace{1.75in}

\part[1] Derive and draw the CDF for $X$.
\vspace{2in}
\newpage
\part[1] Derive a formula for the median of $X$ in terms of $\lambda$. This is also called the \emph{half-life}, which measures the amount of time needed for a sample of radioactive material to degrade by half or for some chemical to degrade or be absorbed by half. Bonus question: what is the half-life of caffeine?
\vspace{1.2in}
\end{parts}
\addpoints


\question[2] Suppose a particular kind of radioactive element has a half-life of 1 year. Find

 \noaddpoints
\begin{parts}
\part[1] The probability that an atom of this type survives for at least 5 years.
\vspace{1.2in}
\part[1] The time at which a sample of this element decays to 10\% of its original purity.
\vspace{1.2in}
\end{parts}
\addpoints

\question[2] Suppose the length $L$ of a phone call is exponentially distributed with $\mu=10$ minutes.
 \noaddpoints
\begin{parts}
\part[1] Compute $P(L\geq 20)$
\vspace{1.2in}
\part[1] Compute $P(8\leq L \leq 22)$
\vspace{1.2in}
\end{parts}
\addpoints

\newpage

\question[3] Measurements on the mass of a metal part produced at a factory are IID with $\mu=12$ grams and $\sigma=1.1$ grams.
 \noaddpoints
\begin{parts}
\part[1] Find the chance that a single measurement is between 11.8 and 12.2 grams, assuming the mass varies according to a normal distribution.
\vspace{1.5in}
\part[1] Estimate the chance that the average of 100 measurements is between 11.8 and 12.2 grams. Answer this is as well: Is it necessary to assume that each measurement is normally distributed?
\vspace{1.7in}
\end{parts}
\addpoints


\end{questions}

\end{document}
