\documentclass[addpoints,12pt]{exam}
\usepackage{fullpage,xypic,mathrsfs}
\usepackage{amsmath, amsthm, amssymb, latexsym}
\usepackage{enumerate}
\usepackage{hyperref}
%\usepackage[sans]{dsfont}

\usepackage{graphicx}

\usepackage[margin=0.7in]{geometry}

\usepackage{bm}
\newcommand{\mbf}[1]{\boldsymbol{\mathbf{#1}}}

\title{\vspace{-1in} AMAT 362---Work Sheet 14}
\date{Due: March 28th, 2022. Worth \numpoints\ points.}
\author{Dr.~Justin M. Curry}


%\input{shortcuts}

\begin{document}
\maketitle

% \begin{center}
% \fbox{\fbox{\parbox{6in}{\centering
% \textbf{Open Book. Open Notes. Open Friends.}}}}
% \end{center}
% \vspace{0.2in}
\makebox[.9\textwidth]{Name:\enspace\hrulefill}
% % \vspace{0.2in}
% % \makebox[\textwidth]{Instructor’s name: Dr. Justin Curry}



Most of these questions are drawn from Grinstead and Snell's Introduction to Probability:

\url{https://math.dartmouth.edu/~prob/prob/prob.pdf}

\begin{questions}
\question \emph{Not Graded. We'll do this in class.} 
Let $S_{100}$ be the number of heads that turn up in 100 tosses of a fair coin. Use
the Central Limit Theorem to estimate
\noaddpoints
\begin{parts}

\part $P(S_{100} \leq 45)$
\vspace{1in}
\part $P(45 < S_{100} < 55)$
\vspace{1in}
\part $P(S_{100} > 63)$
\vspace{1in}
\part $P(S_{100} < 57)$
\vspace{1in}
\end{parts}

\question[2] A fair coin is flipped 400 times.
Determine the number $x$ such that the probability that the number of heads is between $200 - x$ and $200 + x$ is approximately .80.

\vspace{1.5in}

\question[2] A noodle machine in Spumoni’s spaghetti factory makes about 5 percent defective noodles even when properly adjusted. The noodles are then packed
in crates containing 1900 noodles each. A crate is examined and found to
contain 115 defective noodles. What is the approximate probability of finding
at least this many defective noodles if the machine is properly adjusted?

\vspace{1.7in}

\question[4] A restaurant feeds 400 customers per day. On the average 20 percent of the
customers order apple pie.
\noaddpoints
\begin{parts}

\part[2] Give a range (called a 95 percent confidence interval) for the number of
pieces of apple pie ordered on a given day such that you can be 95 percent
sure that the actual number will fall in this range.

\vspace{1.5in}

\part[2] How many customers must the restaurant have, on the average, to be at
least 95 percent sure that the number of customers ordering pie on that
day falls in the 19 to 21 percent range?

\vspace{1.5in}

\end{parts}
\addpoints

\question[2] Once upon a time, there were two railway trains competing for the passenger
traffic of 1000 people leaving from Chicago at the same hour and going to Los
Angeles. Assume that passengers are equally likely to choose each train. How
many seats must a train have to assure a probability of .99 or better of having
a seat for each passenger?

\vspace{1.5in}

\question[2] A die is rolled 24 times. Use the Central Limit Theorem to estimate the
probability that

\noaddpoints
\begin{parts}
\part[1] the sum is greater than 84.

\vspace{1in}

\part[1] the sum is equal to 84.

\vspace{1.2in}

\end{parts}
\addpoints


\question[2] A random walker starts at 0 on the x-axis and at each time unit moves 1
step to the right or 1 step to the left with probability 1/2. Estimate the
probability that, after 100 steps, the walker is more than 10 steps from the
starting position.

\vspace{1.5in}

\question[4] A piece of rope is made up of 100 strands. Assume that the breaking strength
of the rope is the sum of the breaking strengths of the individual strands.
Assume further that this sum may be considered to be the sum of an independent trials process with 100 experiments each having expected value of 10
pounds and standard deviation 1. Find the approximate probability that the
rope will support a weight
\noaddpoints
\begin{parts}
\part[2] of 1000 pounds.

\vspace{1.5in}

\part[2] of 970 pounds.

\vspace{1.2in}

\end{parts}
\addpoints


\question[4] This is a question where we compare bounds produced by Chebyshev's inequality and the central limit theorem. You may find the following information useful: for the uniform distribution on $\{1,2,\ldots, n\}$ the first and second moments are $(n+1)/2$ and $(n+1)(2n+1)/6$.

\noaddpoints
\begin{parts}
\part[2] Let $A_n$ be the average face value after rolling my 12-sided die $n$ times. Using Chebyshev's inequality, how large does $n$ need to be to make $Pr(|A_n-\mu|\geq 1)\leq .05$?

\vspace{2in}

\part[2] Repeat Part A, but by using the central limit theorem.

\vspace{2in}

\end{parts}
\addpoints

\end{questions}

\end{document}
