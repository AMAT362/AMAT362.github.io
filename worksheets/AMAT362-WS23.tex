\documentclass[addpoints,12pt]{exam}
\usepackage{fullpage,xypic,mathrsfs}
\usepackage{amsmath, amsthm, amssymb, latexsym}
\usepackage{enumerate}
\usepackage{hyperref}
%\usepackage[sans]{dsfont}

\usepackage{graphicx}

\usepackage[margin=0.7in]{geometry}

\usepackage{bm}
\newcommand{\mbf}[1]{\boldsymbol{\mathbf{#1}}}

\title{\vspace{-1in} MATH 362---Work Sheet 23}
\date{Saturday May 8, 2021}
\author{Dr.~Justin M. Curry}


%\input{shortcuts}

\begin{document}
\maketitle

% \begin{center}
% \fbox{\fbox{\parbox{6in}{\centering
% \textbf{Open Book. Open Notes. Open Friends.}}}}
% \end{center}
% \vspace{0.2in}
\makebox[.9\textwidth]{Name:\enspace\hrulefill}
% % \vspace{0.2in}
% % \makebox[\textwidth]{Instructor’s name: Dr. Justin Curry}




\begin{questions}

\question[1] A test for a disease $D$ is 95\% accurate, but disease $D$ is only found in 1\% of the population. If a person tests positive for disease $D$, what's the probability that they actually have the disease?

\vspace{1.7in}

%We will focus on problems from the last worksheet, but here are a few ``higher dimensional'' situations worth considering.
\question[5] Suppose there are 11 urns and inside each urn there are 10 balls. In the first urn there are 10 red balls. In the second urn there are 9 red balls and 1 blue ball. In the third urn there are 8 red balls and 2 blue balls. In the fourth urn there are 7 reds and 3 blues. This pattern continues to the eleventh urn, which has 10 blue balls and no red balls.

Now pick one of the eleven urns uniformly at random. Denote the urn number by $U$. Now draw $N$ balls with replacement from the urn. Record the number of blue balls you draw out at as another random variable $B$.

\noaddpoints
\begin{parts}

\part[1] Write down the joint PMF of $U$ and $B$, using equations and familiar distributions. Do not calculate the actual entries.
\vspace{1.5in}
\part[2] Suppose that $N=10$ and $B=3$, what urn number do you believe does the best job of explaining the data? Explain why.
\vspace{2in}

\part[2] Compute $E(B)$.
\vspace{1.7in}

\end{parts}
\addpoints

\question[8] Suppose I toss three coins. Some of them land heads and some of them land tails. Those that land tails I toss again. Let $X$ denote the number of heads after the first tossing and let $Y$ denote the number of heads after the second tossing, which includes the heads from the first tossing that you did not toss again. Observe that $X$ and $Y$ are RVs such that $0\leq X \leq Y \leq 3$.

Write out the distribution tables and sketch the histograms for the following distributions:

\noaddpoints
\begin{parts}

\part[1] The distribution of $X$;
\vspace{1.5in}
\part[1] The conditional distribution of $Y$ given $X=x$ for $x=0,1,2,3$;
\vspace{2in}

\part[2] The joint distribution table of $X$ and $Y$ (no need to sketch the histogram in this case);
\vspace{1.7in}

\newpage
\part[1] The distribution of $Y$;
\vspace{1.5in}

\part[2] The conditional distribution of $X$ given $Y=0,1,2,3$;
\vspace{1.7in}

\part[1] What is the best guess of the value of $X$ given $Y=y$ for $y=0,1,2,3$? That is, for each $y$, choose $x$ that maximizes $P(X=x\mid Y=y)$.
\vspace{1.5in}
\end{parts}
\addpoints

\question[4] In a particular town 10\% of families have no children, 20\% have one child, 40\% have two children, 20\% have three children, and 10\% have four children. Let $T$ denote the total number of children in a randomly selected family from this town. Let $G$ denote the number of girls in a randomly selected family. Assume that children are equally likely to identify as boys or girls. 

\noaddpoints
\begin{parts}

\part[3] Find the distribution for $G$. Display your answer in a table and sketch the histogram for this distribution.

\part[1] Compute $E(G)$.


\end{parts}
\addpoints

\end{questions}

\end{document}
