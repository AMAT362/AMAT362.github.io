\documentclass[addpoints,12pt]{exam}
\usepackage{fullpage,xypic,mathrsfs}
\usepackage{amsmath, amsthm, amssymb, latexsym}
\usepackage{enumerate}
\usepackage{hyperref}
%\usepackage[sans]{dsfont}

\usepackage[margin=0.7in]{geometry}

\usepackage{bm}
\newcommand{\mbf}[1]{\boldsymbol{\mathbf{#1}}}

\title{\vspace{-1in} MATH 362---Work Sheet 03}
\date{Due on February 13, 2021}
\author{Dr.~Justin M. Curry}


%\input{shortcuts}

\begin{document}
\maketitle

% \begin{center}
% \fbox{\fbox{\parbox{6in}{\centering
% \textbf{Open Book. Open Notes. Open Friends.}}}}
% \end{center}
% \vspace{0.2in}
\makebox[.9\textwidth]{Name:\enspace\hrulefill}
% % \vspace{0.2in}
% % \makebox[\textwidth]{Instructor’s name: Dr. Justin Curry}

\begin{questions}

\question[1] Suppose I have $N$ students and I go around and ask everyone their birthdays. What is the size of the sample space $\Omega$ in this experiment? 

\question[1] Continuing Question 1, If $B$ is the event that no one has the same birthday. Describe in words what $B^c$ represents.

\question[2] Since asking people their birthdays can be a little personal. Instead imagine that I ask everyone what their astrological sign (\url{https://en.wikipedia.org/wiki/Astrological_sign}) is. Assuming there are 40 people in my class. What's the probability that at least three people have the same sign? Explain your answer.

\question[1] What's the difference between a Tarot reading and being dealt a five card hand?

\question[3] In a state lottery, 5 distinct numbers are drawn from the numbers 1,2,\ldots,40 uniformly at random.
\noaddpoints
\begin{parts}
\part[1] Describe a sample space $\Omega$ and a probability measure $P$ to model this experiment.

%\vspace{1.5in}

\part[2] What is the probability that out of five picked numbers exactly three will be even?

%\vspace{1.5in}

\end{parts}

\question[2] Suppose that a bag of scrabble tiles contains 5 \texttt{E}'s, 4 \texttt{A}'s, 3 \texttt{N}'s, and 2 \texttt{B}'s. Suppose I draw 4 tiles from the bag without replacement uniformly at random. Let $C$ be the event that I draw two \texttt{E}'s, one \texttt{A} and one \texttt{N}.

\noaddpoints
\begin{parts}
\part[1] Compute $P(C)$ by imagining that the tiles are drawn one by one as an ordered sample.

%\vspace{1.5in}

\part[1] Compute $P(C)$ by imagining that the tiles are drawn all at once as an unordered sample.

%\vspace{1.5in}

\end{parts}

\question[1] What's the probability of a full house?
\end{questions}

\end{document}
