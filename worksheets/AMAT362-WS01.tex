\documentclass[addpoints,12pt]{exam}
\usepackage{fullpage,xypic,mathrsfs}
\usepackage{amsmath, amsthm, amssymb, latexsym}
\usepackage{enumerate}
\usepackage{hyperref}
%\usepackage[sans]{dsfont}

\usepackage[margin=0.7in]{geometry}

\usepackage{bm}
\newcommand{\mbf}[1]{\boldsymbol{\mathbf{#1}}}

\title{\vspace{-1in} AMAT 362 Lecture 1 Worksheet}
\date{February 1, 2021}
\author{Dr.~Justin M. Curry}

%\input{shortcuts}

\begin{document}
\maketitle

% \begin{center}
% \textbf{To be turned in before Exam 1}
% \end{center}
% \vspace{0.2in}
%\makebox[.9\textwidth]{Name:\enspace\hrulefill}
% % \vspace{0.2in}
% % \makebox[\textwidth]{Instructor’s name: Dr. Justin Curry}

\begin{questions}

\question[2] What's the probability of a 4 of a kind, assuming you're dealt a 5 card hand, uniformly at random, from a standard 52 card deck?

%\vspace{1in}

\question[1] How much money do you need to make to be in the top 1\% of income earners in the United States? What about the top 1\% of New Yorkers? Do you think there is such a thing as a ``fair'' distribution of incomes? What does that look like?

%\vspace{1in}

\question[1] What's the probability of life on Mars? What does this question illustrate about the different meanings of the term ``probability''?

%
\question[1] How many people should you date before settling down on ``the one,'' assuming that's something you want to do?

%\vspace{1in}

\question[6] Translate each of the following symbolic expressions into English statements:

\noaddpoints
\begin{parts}
  \part[1] $x\in A$
  \part[1] $A\subseteq B$
  \part[1] $A^c$
  \part[1] $R\times S$
  \part[1] $|A|$
  \part[1] How do you interpret $\varnothing$ and $\Omega$ in probability?
\end{parts}
\addpoints

\question[2] \textbf{In Class:} State De Morgan's Laws. What does this have to do with the star battle problem? \url{https://krazydad.com/tablet/starbattle/}

%\vspace{1in}

\question[2] Consider the set $\Omega$, the union operation $A \cup B$ of subsets of $\Omega$ and the intersection operation $A \cap B$ on $\Omega$. What does it mean to say that $\cup$ and $\cap$ are associative and symmetric? How does the union operation ``distribute over'' the intersection operation?

%\vspace{1in}

\question[5] Suppose we have a deck of 20 cards, 10 are red and 10 are blue. Each of the blue cards has a unique number between 1 and 10. Each of the red cards has a unique number also between 1 and 10.

\noaddpoints
\begin{parts}
\part[1] Describe the sample space $\Omega$ as a Cartesian product.

%\vspace{1in}

\part[1] Consider the following events:
\begin{itemize}
  \item Let $A$ be the event that a card drawn has an even number on it.
  \item Let $B$ be the event that a blue card is drawn.
  \item Let $C$ be the event that a card with a number (strictly) less than $5$ is drawn.
\end{itemize}
What are the sizes of $A$, $B$, and $C$?

%\vspace{1in}

\part[2] Describe the events $A\cup B \cup C$ and $A^c \cap B^c \cap C^c$.

%\vspace{2.5in}

\part[1] What are the number of outcomes in each of the events in part (c)?
\end{parts}

\end{questions}

\end{document}
