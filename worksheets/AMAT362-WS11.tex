\documentclass[addpoints,12pt]{exam}
\usepackage{fullpage,xypic,mathrsfs}
\usepackage{amsmath, amsthm, amssymb, latexsym}
\usepackage{enumerate}
\usepackage{hyperref}
%\usepackage[sans]{dsfont}

\usepackage{graphicx}

\usepackage[margin=0.7in]{geometry}

\usepackage{bm}
\newcommand{\mbf}[1]{\boldsymbol{\mathbf{#1}}}

\title{\vspace{-1in} AMAT 362---Work Sheet 11}
\date{Due: March 9th, 2022. Worth \numpoints\ points.}
\author{Dr.~Justin M. Curry}


%\input{shortcuts}

\begin{document}
\maketitle

% \begin{center}
% \fbox{\fbox{\parbox{6in}{\centering
% \textbf{Open Book. Open Notes. Open Friends.}}}}
% \end{center}
% \vspace{0.2in}
\makebox[.9\textwidth]{Name:\enspace\hrulefill}
% % \vspace{0.2in}
% % \makebox[\textwidth]{Instructor’s name: Dr. Justin Curry}

\begin{questions}

\question[1] What is the expected number of sixes appear on 3 die rolls? What is the expected number of odd numbers?

\vspace{.7in}

\question[1] Let $X$ be the number of spaces in 7 cards dealt from a well shuffled standard 52 card deck. What is $E(X)$?

\vspace{1in}

\question[6] A fair six-sided die is rolled 2 times. Let $L$ denote the number of times a value strictly less than 4 is rolled. Let $M$ denote the number of times a 4 is rolled. Let $H$ denote the number of times a 5 or 6 is rolled.

\noaddpoints
\begin{parts}
\part[2] Write down the joint PMF for $(L,M,H)$.
\vspace{2in}

\part[2] What is the distribution for $L$? Write out the PMF for $L$.
\vspace{1.2in}

\part[2] What is the distribution for $L+H$?
\vspace{1.5in}
\end{parts}
\addpoints

\newpage

\question[1] Suppose $E(X^2)=3$, $E(Y^2)=4$, $E(XY)=2$. Find $E((X+Y)^2)$.

\vspace{1in}

\question[2] In a circuit containing $n$ switches, the $i^{th}$ switch is closed with probability $p_i$. Let $X$ denote the total number of switches that are closed. What is $E(X)$?

\vspace{1in}

\question[5] There are 100 prize tickets among 1000 tickets in a lottery. 
\noaddpoints
\begin{parts}
\part[1] What is the expected number of prize tickets you will get if you buy 3 tickets?
\vspace{.7in}

\part[2] What is a simple upper bound for the probability that you will win at least one prize?
\vspace{1.2in}

\part[2] Calculate the actual probability. Why is this answer so close to the actual answer?
\vspace{1in}
\end{parts}
\addpoints

\end{questions}

\end{document}
