\documentclass[addpoints,12pt]{exam}
\usepackage{fullpage,xypic,mathrsfs}
\usepackage{amsmath, amsthm, amssymb, latexsym}
\usepackage{enumerate}
\usepackage{hyperref}
%\usepackage[sans]{dsfont}

\usepackage{graphicx}

\usepackage[margin=0.7in]{geometry}

\usepackage{bm}
\newcommand{\mbf}[1]{\boldsymbol{\mathbf{#1}}}

\title{\vspace{-1in} MATH 362---Work Sheet 06}
\date{Due on Monday February 22nd, 2021}
\author{Dr.~Justin M. Curry}


%\input{shortcuts}

\begin{document}
\maketitle

% \begin{center}
% \fbox{\fbox{\parbox{6in}{\centering
% \textbf{Open Book. Open Notes. Open Friends.}}}}
% \end{center}
% \vspace{0.2in}
\makebox[.9\textwidth]{Name:\enspace\hrulefill}
% % \vspace{0.2in}
% % \makebox[\textwidth]{Instructor’s name: Dr. Justin Curry}

\begin{questions}
\question[5]
Assume the following (overly simplified) statements are true: 
\begin{itemize}
	\item In the US, 50\% of the population are men and 50\% of the population are women. 
	\item 10\% of the population in the US practices yoga.
	\item Of yoga practitioners, 72\% are women and 28\% are men.
\end{itemize}
\noaddpoints
\begin{parts}
\part[1] Suppose that I'm told a person practices yoga, what is the probability that they are a woman?

\vspace{1in}

\part[1]  Are the events of being a woman and practicing yoga independent?

\vspace{1in}

\part[2]  Suppose I randomly select a woman from the US population. What is the probability that they practice yoga?

\vspace{1in}

\part[1]  What's the probability of a yoga class with five people in it consisting of all men?

\vspace{1in}

\end{parts}
\addpoints

\newpage
\question[4]
\emph{Conditional Independence and Testing Twice for a Disease:} For two events $E_1$ and $E_2$ we sometimes write
\[
	P(E_1 \cap E_2) = P(E_1 \text{ and } E_2) \qquad \text{as} \qquad P(E_1E_2)
\]
in order to save space.
$E_1$ and $E_2$ are independent if $P(E_1 E_2)=P(E_1) P(E_2)$.
Furthermore, we say $E_1$ and $E_2$ are \emph{conditionally independent given $B$} if 
\[
	P(E_1 E_2 \mid B)= P(E_1\mid B) P(E_2 \mid B).
\]
An example of conditional independence occurs when we re-run a test for a disease $D$.
The probability of two positive tests assuming you have the disease can be computed as 
\[
	P(++\mid D)=P(+\mid D) P(+\mid D) = P(+ \mid D)^2
\]
because each test is performed independently, even assuming you have the disease $D$.

For this question, assume the following is true:
\begin{itemize}
	\item The probability that a randomly selected person has disease $D$ is $0.5\%$.
	\item $P(+\mid D)=96\%$
	\item $P(+ \mid D^c) = 2\%$
\end{itemize}
\noaddpoints
\begin{parts}
\part[2] Assuming someone tests positive for the disease $D$, what's the probability that they actually have disease $D$?

\vspace{1.5in}

\part[2] Assuming someone tests positive \emph{twice} for a disease, what's the probability that they actually have disease $D$?
\vspace{1.75in}

\end{parts}
\addpoints

\question[2] Suppose I have two urns, $U_1$ has two red balls and one white ball and $U_2$ has two red balls and two white balls.
I select an urn uniformly at random, and draw out a red ball. What's the probability that I selected $U_1$?

\vspace{1in}

\question[1] Suppose that $P(A)=1/3$ and $P(B)=1/3$ and $P(A B^c)=2/9$. Are $A$ and $B$ independent?

\vspace{1in}

\begin{figure}[h]
\centering
\includegraphics[width=3in]{component-in-series}
\caption{Two Components in Series}
\end{figure}

\question[2] Suppose that in an electrical device with two components in \emph{series}, the device works only if both components $C_1$ and $C_2$ work. Say the probability of each component failing on a given day is $10\%$ and $5\%$ respectively. What's the probability that the device works on a given day?

\vspace{1.2in}

\begin{figure}[h]
\centering
\includegraphics[width=2in]{component-in-parallelv2}
\caption{Two Components in Parallel}

\end{figure}

\question[2] Suppose that in an electrical device with two components in \emph{parallel}, the device works only if either component $C_1$ or $C_2$ works. Say the probability of each component failing on a given day is $10\%$ and $5\%$ respectively. What's the probability that the device works on a given day?

\end{questions}

\end{document}
